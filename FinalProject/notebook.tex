
% Default to the notebook output style

    


% Inherit from the specified cell style.




    
\documentclass[11pt]{article}

    
    
    \usepackage[T1]{fontenc}
    % Nicer default font (+ math font) than Computer Modern for most use cases
    \usepackage{mathpazo}

    % Basic figure setup, for now with no caption control since it's done
    % automatically by Pandoc (which extracts ![](path) syntax from Markdown).
    \usepackage{graphicx}
    % We will generate all images so they have a width \maxwidth. This means
    % that they will get their normal width if they fit onto the page, but
    % are scaled down if they would overflow the margins.
    \makeatletter
    \def\maxwidth{\ifdim\Gin@nat@width>\linewidth\linewidth
    \else\Gin@nat@width\fi}
    \makeatother
    \let\Oldincludegraphics\includegraphics
    % Set max figure width to be 80% of text width, for now hardcoded.
    \renewcommand{\includegraphics}[1]{\Oldincludegraphics[width=.8\maxwidth]{#1}}
    % Ensure that by default, figures have no caption (until we provide a
    % proper Figure object with a Caption API and a way to capture that
    % in the conversion process - todo).
    \usepackage{caption}
    \DeclareCaptionLabelFormat{nolabel}{}
    \captionsetup{labelformat=nolabel}

    \usepackage{adjustbox} % Used to constrain images to a maximum size 
    \usepackage{xcolor} % Allow colors to be defined
    \usepackage{enumerate} % Needed for markdown enumerations to work
    \usepackage{geometry} % Used to adjust the document margins
    \usepackage{amsmath} % Equations
    \usepackage{amssymb} % Equations
    \usepackage{textcomp} % defines textquotesingle
    % Hack from http://tex.stackexchange.com/a/47451/13684:
    \AtBeginDocument{%
        \def\PYZsq{\textquotesingle}% Upright quotes in Pygmentized code
    }
    \usepackage{upquote} % Upright quotes for verbatim code
    \usepackage{eurosym} % defines \euro
    \usepackage[mathletters]{ucs} % Extended unicode (utf-8) support
    \usepackage[utf8x]{inputenc} % Allow utf-8 characters in the tex document
    \usepackage{fancyvrb} % verbatim replacement that allows latex
    \usepackage{grffile} % extends the file name processing of package graphics 
                         % to support a larger range 
    % The hyperref package gives us a pdf with properly built
    % internal navigation ('pdf bookmarks' for the table of contents,
    % internal cross-reference links, web links for URLs, etc.)
    \usepackage{hyperref}
    \usepackage{longtable} % longtable support required by pandoc >1.10
    \usepackage{booktabs}  % table support for pandoc > 1.12.2
    \usepackage[inline]{enumitem} % IRkernel/repr support (it uses the enumerate* environment)
    \usepackage[normalem]{ulem} % ulem is needed to support strikethroughs (\sout)
                                % normalem makes italics be italics, not underlines
    

    
    
    % Colors for the hyperref package
    \definecolor{urlcolor}{rgb}{0,.145,.698}
    \definecolor{linkcolor}{rgb}{.71,0.21,0.01}
    \definecolor{citecolor}{rgb}{.12,.54,.11}

    % ANSI colors
    \definecolor{ansi-black}{HTML}{3E424D}
    \definecolor{ansi-black-intense}{HTML}{282C36}
    \definecolor{ansi-red}{HTML}{E75C58}
    \definecolor{ansi-red-intense}{HTML}{B22B31}
    \definecolor{ansi-green}{HTML}{00A250}
    \definecolor{ansi-green-intense}{HTML}{007427}
    \definecolor{ansi-yellow}{HTML}{DDB62B}
    \definecolor{ansi-yellow-intense}{HTML}{B27D12}
    \definecolor{ansi-blue}{HTML}{208FFB}
    \definecolor{ansi-blue-intense}{HTML}{0065CA}
    \definecolor{ansi-magenta}{HTML}{D160C4}
    \definecolor{ansi-magenta-intense}{HTML}{A03196}
    \definecolor{ansi-cyan}{HTML}{60C6C8}
    \definecolor{ansi-cyan-intense}{HTML}{258F8F}
    \definecolor{ansi-white}{HTML}{C5C1B4}
    \definecolor{ansi-white-intense}{HTML}{A1A6B2}

    % commands and environments needed by pandoc snippets
    % extracted from the output of `pandoc -s`
    \providecommand{\tightlist}{%
      \setlength{\itemsep}{0pt}\setlength{\parskip}{0pt}}
    \DefineVerbatimEnvironment{Highlighting}{Verbatim}{commandchars=\\\{\}}
    % Add ',fontsize=\small' for more characters per line
    \newenvironment{Shaded}{}{}
    \newcommand{\KeywordTok}[1]{\textcolor[rgb]{0.00,0.44,0.13}{\textbf{{#1}}}}
    \newcommand{\DataTypeTok}[1]{\textcolor[rgb]{0.56,0.13,0.00}{{#1}}}
    \newcommand{\DecValTok}[1]{\textcolor[rgb]{0.25,0.63,0.44}{{#1}}}
    \newcommand{\BaseNTok}[1]{\textcolor[rgb]{0.25,0.63,0.44}{{#1}}}
    \newcommand{\FloatTok}[1]{\textcolor[rgb]{0.25,0.63,0.44}{{#1}}}
    \newcommand{\CharTok}[1]{\textcolor[rgb]{0.25,0.44,0.63}{{#1}}}
    \newcommand{\StringTok}[1]{\textcolor[rgb]{0.25,0.44,0.63}{{#1}}}
    \newcommand{\CommentTok}[1]{\textcolor[rgb]{0.38,0.63,0.69}{\textit{{#1}}}}
    \newcommand{\OtherTok}[1]{\textcolor[rgb]{0.00,0.44,0.13}{{#1}}}
    \newcommand{\AlertTok}[1]{\textcolor[rgb]{1.00,0.00,0.00}{\textbf{{#1}}}}
    \newcommand{\FunctionTok}[1]{\textcolor[rgb]{0.02,0.16,0.49}{{#1}}}
    \newcommand{\RegionMarkerTok}[1]{{#1}}
    \newcommand{\ErrorTok}[1]{\textcolor[rgb]{1.00,0.00,0.00}{\textbf{{#1}}}}
    \newcommand{\NormalTok}[1]{{#1}}
    
    % Additional commands for more recent versions of Pandoc
    \newcommand{\ConstantTok}[1]{\textcolor[rgb]{0.53,0.00,0.00}{{#1}}}
    \newcommand{\SpecialCharTok}[1]{\textcolor[rgb]{0.25,0.44,0.63}{{#1}}}
    \newcommand{\VerbatimStringTok}[1]{\textcolor[rgb]{0.25,0.44,0.63}{{#1}}}
    \newcommand{\SpecialStringTok}[1]{\textcolor[rgb]{0.73,0.40,0.53}{{#1}}}
    \newcommand{\ImportTok}[1]{{#1}}
    \newcommand{\DocumentationTok}[1]{\textcolor[rgb]{0.73,0.13,0.13}{\textit{{#1}}}}
    \newcommand{\AnnotationTok}[1]{\textcolor[rgb]{0.38,0.63,0.69}{\textbf{\textit{{#1}}}}}
    \newcommand{\CommentVarTok}[1]{\textcolor[rgb]{0.38,0.63,0.69}{\textbf{\textit{{#1}}}}}
    \newcommand{\VariableTok}[1]{\textcolor[rgb]{0.10,0.09,0.49}{{#1}}}
    \newcommand{\ControlFlowTok}[1]{\textcolor[rgb]{0.00,0.44,0.13}{\textbf{{#1}}}}
    \newcommand{\OperatorTok}[1]{\textcolor[rgb]{0.40,0.40,0.40}{{#1}}}
    \newcommand{\BuiltInTok}[1]{{#1}}
    \newcommand{\ExtensionTok}[1]{{#1}}
    \newcommand{\PreprocessorTok}[1]{\textcolor[rgb]{0.74,0.48,0.00}{{#1}}}
    \newcommand{\AttributeTok}[1]{\textcolor[rgb]{0.49,0.56,0.16}{{#1}}}
    \newcommand{\InformationTok}[1]{\textcolor[rgb]{0.38,0.63,0.69}{\textbf{\textit{{#1}}}}}
    \newcommand{\WarningTok}[1]{\textcolor[rgb]{0.38,0.63,0.69}{\textbf{\textit{{#1}}}}}
    
    
    % Define a nice break command that doesn't care if a line doesn't already
    % exist.
    \def\br{\hspace*{\fill} \\* }
    % Math Jax compatability definitions
    \def\gt{>}
    \def\lt{<}
    % Document parameters
    \title{Exponential Distribution (epidemiology)}
    
    
    

    % Pygments definitions
    
\makeatletter
\def\PY@reset{\let\PY@it=\relax \let\PY@bf=\relax%
    \let\PY@ul=\relax \let\PY@tc=\relax%
    \let\PY@bc=\relax \let\PY@ff=\relax}
\def\PY@tok#1{\csname PY@tok@#1\endcsname}
\def\PY@toks#1+{\ifx\relax#1\empty\else%
    \PY@tok{#1}\expandafter\PY@toks\fi}
\def\PY@do#1{\PY@bc{\PY@tc{\PY@ul{%
    \PY@it{\PY@bf{\PY@ff{#1}}}}}}}
\def\PY#1#2{\PY@reset\PY@toks#1+\relax+\PY@do{#2}}

\expandafter\def\csname PY@tok@w\endcsname{\def\PY@tc##1{\textcolor[rgb]{0.73,0.73,0.73}{##1}}}
\expandafter\def\csname PY@tok@c\endcsname{\let\PY@it=\textit\def\PY@tc##1{\textcolor[rgb]{0.25,0.50,0.50}{##1}}}
\expandafter\def\csname PY@tok@cp\endcsname{\def\PY@tc##1{\textcolor[rgb]{0.74,0.48,0.00}{##1}}}
\expandafter\def\csname PY@tok@k\endcsname{\let\PY@bf=\textbf\def\PY@tc##1{\textcolor[rgb]{0.00,0.50,0.00}{##1}}}
\expandafter\def\csname PY@tok@kp\endcsname{\def\PY@tc##1{\textcolor[rgb]{0.00,0.50,0.00}{##1}}}
\expandafter\def\csname PY@tok@kt\endcsname{\def\PY@tc##1{\textcolor[rgb]{0.69,0.00,0.25}{##1}}}
\expandafter\def\csname PY@tok@o\endcsname{\def\PY@tc##1{\textcolor[rgb]{0.40,0.40,0.40}{##1}}}
\expandafter\def\csname PY@tok@ow\endcsname{\let\PY@bf=\textbf\def\PY@tc##1{\textcolor[rgb]{0.67,0.13,1.00}{##1}}}
\expandafter\def\csname PY@tok@nb\endcsname{\def\PY@tc##1{\textcolor[rgb]{0.00,0.50,0.00}{##1}}}
\expandafter\def\csname PY@tok@nf\endcsname{\def\PY@tc##1{\textcolor[rgb]{0.00,0.00,1.00}{##1}}}
\expandafter\def\csname PY@tok@nc\endcsname{\let\PY@bf=\textbf\def\PY@tc##1{\textcolor[rgb]{0.00,0.00,1.00}{##1}}}
\expandafter\def\csname PY@tok@nn\endcsname{\let\PY@bf=\textbf\def\PY@tc##1{\textcolor[rgb]{0.00,0.00,1.00}{##1}}}
\expandafter\def\csname PY@tok@ne\endcsname{\let\PY@bf=\textbf\def\PY@tc##1{\textcolor[rgb]{0.82,0.25,0.23}{##1}}}
\expandafter\def\csname PY@tok@nv\endcsname{\def\PY@tc##1{\textcolor[rgb]{0.10,0.09,0.49}{##1}}}
\expandafter\def\csname PY@tok@no\endcsname{\def\PY@tc##1{\textcolor[rgb]{0.53,0.00,0.00}{##1}}}
\expandafter\def\csname PY@tok@nl\endcsname{\def\PY@tc##1{\textcolor[rgb]{0.63,0.63,0.00}{##1}}}
\expandafter\def\csname PY@tok@ni\endcsname{\let\PY@bf=\textbf\def\PY@tc##1{\textcolor[rgb]{0.60,0.60,0.60}{##1}}}
\expandafter\def\csname PY@tok@na\endcsname{\def\PY@tc##1{\textcolor[rgb]{0.49,0.56,0.16}{##1}}}
\expandafter\def\csname PY@tok@nt\endcsname{\let\PY@bf=\textbf\def\PY@tc##1{\textcolor[rgb]{0.00,0.50,0.00}{##1}}}
\expandafter\def\csname PY@tok@nd\endcsname{\def\PY@tc##1{\textcolor[rgb]{0.67,0.13,1.00}{##1}}}
\expandafter\def\csname PY@tok@s\endcsname{\def\PY@tc##1{\textcolor[rgb]{0.73,0.13,0.13}{##1}}}
\expandafter\def\csname PY@tok@sd\endcsname{\let\PY@it=\textit\def\PY@tc##1{\textcolor[rgb]{0.73,0.13,0.13}{##1}}}
\expandafter\def\csname PY@tok@si\endcsname{\let\PY@bf=\textbf\def\PY@tc##1{\textcolor[rgb]{0.73,0.40,0.53}{##1}}}
\expandafter\def\csname PY@tok@se\endcsname{\let\PY@bf=\textbf\def\PY@tc##1{\textcolor[rgb]{0.73,0.40,0.13}{##1}}}
\expandafter\def\csname PY@tok@sr\endcsname{\def\PY@tc##1{\textcolor[rgb]{0.73,0.40,0.53}{##1}}}
\expandafter\def\csname PY@tok@ss\endcsname{\def\PY@tc##1{\textcolor[rgb]{0.10,0.09,0.49}{##1}}}
\expandafter\def\csname PY@tok@sx\endcsname{\def\PY@tc##1{\textcolor[rgb]{0.00,0.50,0.00}{##1}}}
\expandafter\def\csname PY@tok@m\endcsname{\def\PY@tc##1{\textcolor[rgb]{0.40,0.40,0.40}{##1}}}
\expandafter\def\csname PY@tok@gh\endcsname{\let\PY@bf=\textbf\def\PY@tc##1{\textcolor[rgb]{0.00,0.00,0.50}{##1}}}
\expandafter\def\csname PY@tok@gu\endcsname{\let\PY@bf=\textbf\def\PY@tc##1{\textcolor[rgb]{0.50,0.00,0.50}{##1}}}
\expandafter\def\csname PY@tok@gd\endcsname{\def\PY@tc##1{\textcolor[rgb]{0.63,0.00,0.00}{##1}}}
\expandafter\def\csname PY@tok@gi\endcsname{\def\PY@tc##1{\textcolor[rgb]{0.00,0.63,0.00}{##1}}}
\expandafter\def\csname PY@tok@gr\endcsname{\def\PY@tc##1{\textcolor[rgb]{1.00,0.00,0.00}{##1}}}
\expandafter\def\csname PY@tok@ge\endcsname{\let\PY@it=\textit}
\expandafter\def\csname PY@tok@gs\endcsname{\let\PY@bf=\textbf}
\expandafter\def\csname PY@tok@gp\endcsname{\let\PY@bf=\textbf\def\PY@tc##1{\textcolor[rgb]{0.00,0.00,0.50}{##1}}}
\expandafter\def\csname PY@tok@go\endcsname{\def\PY@tc##1{\textcolor[rgb]{0.53,0.53,0.53}{##1}}}
\expandafter\def\csname PY@tok@gt\endcsname{\def\PY@tc##1{\textcolor[rgb]{0.00,0.27,0.87}{##1}}}
\expandafter\def\csname PY@tok@err\endcsname{\def\PY@bc##1{\setlength{\fboxsep}{0pt}\fcolorbox[rgb]{1.00,0.00,0.00}{1,1,1}{\strut ##1}}}
\expandafter\def\csname PY@tok@kc\endcsname{\let\PY@bf=\textbf\def\PY@tc##1{\textcolor[rgb]{0.00,0.50,0.00}{##1}}}
\expandafter\def\csname PY@tok@kd\endcsname{\let\PY@bf=\textbf\def\PY@tc##1{\textcolor[rgb]{0.00,0.50,0.00}{##1}}}
\expandafter\def\csname PY@tok@kn\endcsname{\let\PY@bf=\textbf\def\PY@tc##1{\textcolor[rgb]{0.00,0.50,0.00}{##1}}}
\expandafter\def\csname PY@tok@kr\endcsname{\let\PY@bf=\textbf\def\PY@tc##1{\textcolor[rgb]{0.00,0.50,0.00}{##1}}}
\expandafter\def\csname PY@tok@bp\endcsname{\def\PY@tc##1{\textcolor[rgb]{0.00,0.50,0.00}{##1}}}
\expandafter\def\csname PY@tok@fm\endcsname{\def\PY@tc##1{\textcolor[rgb]{0.00,0.00,1.00}{##1}}}
\expandafter\def\csname PY@tok@vc\endcsname{\def\PY@tc##1{\textcolor[rgb]{0.10,0.09,0.49}{##1}}}
\expandafter\def\csname PY@tok@vg\endcsname{\def\PY@tc##1{\textcolor[rgb]{0.10,0.09,0.49}{##1}}}
\expandafter\def\csname PY@tok@vi\endcsname{\def\PY@tc##1{\textcolor[rgb]{0.10,0.09,0.49}{##1}}}
\expandafter\def\csname PY@tok@vm\endcsname{\def\PY@tc##1{\textcolor[rgb]{0.10,0.09,0.49}{##1}}}
\expandafter\def\csname PY@tok@sa\endcsname{\def\PY@tc##1{\textcolor[rgb]{0.73,0.13,0.13}{##1}}}
\expandafter\def\csname PY@tok@sb\endcsname{\def\PY@tc##1{\textcolor[rgb]{0.73,0.13,0.13}{##1}}}
\expandafter\def\csname PY@tok@sc\endcsname{\def\PY@tc##1{\textcolor[rgb]{0.73,0.13,0.13}{##1}}}
\expandafter\def\csname PY@tok@dl\endcsname{\def\PY@tc##1{\textcolor[rgb]{0.73,0.13,0.13}{##1}}}
\expandafter\def\csname PY@tok@s2\endcsname{\def\PY@tc##1{\textcolor[rgb]{0.73,0.13,0.13}{##1}}}
\expandafter\def\csname PY@tok@sh\endcsname{\def\PY@tc##1{\textcolor[rgb]{0.73,0.13,0.13}{##1}}}
\expandafter\def\csname PY@tok@s1\endcsname{\def\PY@tc##1{\textcolor[rgb]{0.73,0.13,0.13}{##1}}}
\expandafter\def\csname PY@tok@mb\endcsname{\def\PY@tc##1{\textcolor[rgb]{0.40,0.40,0.40}{##1}}}
\expandafter\def\csname PY@tok@mf\endcsname{\def\PY@tc##1{\textcolor[rgb]{0.40,0.40,0.40}{##1}}}
\expandafter\def\csname PY@tok@mh\endcsname{\def\PY@tc##1{\textcolor[rgb]{0.40,0.40,0.40}{##1}}}
\expandafter\def\csname PY@tok@mi\endcsname{\def\PY@tc##1{\textcolor[rgb]{0.40,0.40,0.40}{##1}}}
\expandafter\def\csname PY@tok@il\endcsname{\def\PY@tc##1{\textcolor[rgb]{0.40,0.40,0.40}{##1}}}
\expandafter\def\csname PY@tok@mo\endcsname{\def\PY@tc##1{\textcolor[rgb]{0.40,0.40,0.40}{##1}}}
\expandafter\def\csname PY@tok@ch\endcsname{\let\PY@it=\textit\def\PY@tc##1{\textcolor[rgb]{0.25,0.50,0.50}{##1}}}
\expandafter\def\csname PY@tok@cm\endcsname{\let\PY@it=\textit\def\PY@tc##1{\textcolor[rgb]{0.25,0.50,0.50}{##1}}}
\expandafter\def\csname PY@tok@cpf\endcsname{\let\PY@it=\textit\def\PY@tc##1{\textcolor[rgb]{0.25,0.50,0.50}{##1}}}
\expandafter\def\csname PY@tok@c1\endcsname{\let\PY@it=\textit\def\PY@tc##1{\textcolor[rgb]{0.25,0.50,0.50}{##1}}}
\expandafter\def\csname PY@tok@cs\endcsname{\let\PY@it=\textit\def\PY@tc##1{\textcolor[rgb]{0.25,0.50,0.50}{##1}}}

\def\PYZbs{\char`\\}
\def\PYZus{\char`\_}
\def\PYZob{\char`\{}
\def\PYZcb{\char`\}}
\def\PYZca{\char`\^}
\def\PYZam{\char`\&}
\def\PYZlt{\char`\<}
\def\PYZgt{\char`\>}
\def\PYZsh{\char`\#}
\def\PYZpc{\char`\%}
\def\PYZdl{\char`\$}
\def\PYZhy{\char`\-}
\def\PYZsq{\char`\'}
\def\PYZdq{\char`\"}
\def\PYZti{\char`\~}
% for compatibility with earlier versions
\def\PYZat{@}
\def\PYZlb{[}
\def\PYZrb{]}
\makeatother


    % Exact colors from NB
    \definecolor{incolor}{rgb}{0.0, 0.0, 0.5}
    \definecolor{outcolor}{rgb}{0.545, 0.0, 0.0}



    
    % Prevent overflowing lines due to hard-to-break entities
    \sloppy 
    % Setup hyperref package
    \hypersetup{
      breaklinks=true,  % so long urls are correctly broken across lines
      colorlinks=true,
      urlcolor=urlcolor,
      linkcolor=linkcolor,
      citecolor=citecolor,
      }
    % Slightly bigger margins than the latex defaults
    
    \geometry{verbose,tmargin=1in,bmargin=1in,lmargin=1in,rmargin=1in}
    
    

    \begin{document}
    
    
    \maketitle
    
    

    
    \subsubsection{Timorco}\label{timorco}

\paragraph{Eduardo Vaca A01207563}\label{eduardo-vaca-a01207563}

\paragraph{Raúl Mar A00512318}\label{rauxfal-mar-a00512318}

\paragraph{Javier Rodríguez
A01152572}\label{javier-rodruxedguez-a01152572}

\section{Exponential
distribution(Epidemiology)}\label{exponential-distributionepidemiology}

    \subsubsection{Introduction}\label{introduction}

\begin{verbatim}
    Virus and diseases are big issue for public health because of their ability to propagate and infect new hosts, in order to prevent a rapid expansion and to give the scientist and researchers a leverage against these entities, the field of Continuos Event System simulation comes in. We could define a simulation as "the imitation of the operation of a real-world process or system over time."[1]

    For this project TIMORCO simulated the evolution of 2 diferent Virus strains: The ebola virus which was known for being a quickly expanding epidemia which started in Guinnea and quickly moved to the neighbor countries such as Sierra Leone and Liberia. The second simulation was done on the normal flu which currently kills arround
375,000 people per year, giving us two great test cases for our reasearch.
\end{verbatim}

    \section{Methodology}\label{methodology}

\begin{verbatim}
In order to simulate the virus transmision process we first defined two hosts:
     1. People from 20 to 40 years old
     2. Babies from 0 to 5 years old, this group is more susceptible to get sick and die

Also as we previously mentioned two virus strains were modeled:
     1. Agressive dissease, simmulating ebola
     2. Soft dissease, simmulating a normal flu

The virus lifecycle goes through these 3 following states:
    ![process][img/process.png]

On which we looked for the number of individuals which were the most succeptible to catch the virus and we used the following time based equation for each of the strains to calculate it:
    ![succeptible][img/succeptible.png]
 
Also the current infected population was analyzed:
    ![infected][img/infected.png]

Finally we calculated the recovered users over time. and this was used to figure out the succeptibility of the users to catch the virus again.
   ![recovered][img/recovered.png]


 In order to calculate this we first need to calculate the growth rate(GH) of each of the strains which follow an exponential distribution
    
\end{verbatim}

    \begin{Verbatim}[commandchars=\\\{\}]
{\color{incolor}In [{\color{incolor}1}]:} \PY{c+c1}{\PYZsh{} number of hosts}
        \PY{n}{N\PYZus{}H} \PY{o}{=} \PY{l+m+mi}{2}
        \PY{c+c1}{\PYZsh{} number of virus strains}
        \PY{n}{N\PYZus{}V} \PY{o}{=} \PY{l+m+mi}{2}
\end{Verbatim}


    \#\#\# Generate probability of mortality for the different type of hosts
and strains We set the values depending on the host group and virus
strain (one strain is more aggressive than the other) \#\#\#\# Where:
\emph{MU represents the mortality probability of each of the strains
}BETA represents the probability of catching the virus \emph{RO
represents the probability of catching the virus again after being
already infected, which is also a transistion state between recovered
and succeptible }GAMMA is the probability of recovering from the virus.

    \begin{Verbatim}[commandchars=\\\{\}]
{\color{incolor}In [{\color{incolor}2}]:} \PY{k+kn}{import} \PY{n+nn}{random}
        \PY{c+c1}{\PYZsh{} def no\PYZus{}mortality\PYZus{}prob():}
        \PY{c+c1}{\PYZsh{}     return [random.uniform(0,1) for \PYZus{} in range(N\PYZus{}H)]}
        
        \PY{c+c1}{\PYZsh{} def virus\PYZus{}host\PYZus{}prob():}
        \PY{c+c1}{\PYZsh{}     return [[random.uniform(0,1) for \PYZus{} in range(N\PYZus{}V)] for \PYZus{} in range(N\PYZus{}H)]}
        
        \PY{c+c1}{\PYZsh{} MU = no\PYZus{}mortality\PYZus{}prob()}
        \PY{n}{MU} \PY{o}{=} \PY{p}{[}\PY{l+m+mf}{0.90}\PY{p}{,} \PY{l+m+mf}{0.10}\PY{p}{]} \PY{c+c1}{\PYZsh{} no mortality probability}
        
        \PY{c+c1}{\PYZsh{} BETA = virus\PYZus{}host\PYZus{}prob()}
        \PY{n}{BETA} \PY{o}{=} \PY{p}{[}\PY{p}{[}\PY{l+m+mf}{0.6}\PY{p}{,} \PY{l+m+mf}{0.3}\PY{p}{]}\PY{p}{,} \PY{p}{[}\PY{l+m+mf}{0.9}\PY{p}{,} \PY{l+m+mf}{0.05}\PY{p}{]}\PY{p}{]} \PY{c+c1}{\PYZsh{} probability of catching the virus}
        \PY{c+c1}{\PYZsh{} print(BETA)}
        \PY{c+c1}{\PYZsh{} RO = virus\PYZus{}host\PYZus{}prob()}
        \PY{n}{RO} \PY{o}{=} \PY{p}{[}\PY{p}{[}\PY{l+m+mf}{0.75}\PY{p}{,} \PY{l+m+mf}{0.15}\PY{p}{]}\PY{p}{,} \PY{p}{[}\PY{l+m+mf}{0.95}\PY{p}{,} \PY{l+m+mf}{0.5}\PY{p}{]}\PY{p}{]} \PY{c+c1}{\PYZsh{} probability of catching the virus again after being already infected}
        
        \PY{c+c1}{\PYZsh{} GAMMA = virus\PYZus{}host\PYZus{}prob()}
        \PY{n}{GAMMA} \PY{o}{=} \PY{p}{[}\PY{p}{[}\PY{l+m+mf}{0.90}\PY{p}{,} \PY{l+m+mf}{0.70}\PY{p}{]}\PY{p}{,} \PY{p}{[}\PY{l+m+mf}{0.1}\PY{p}{,} \PY{l+m+mf}{0.4}\PY{p}{]}\PY{p}{]} \PY{c+c1}{\PYZsh{} probability of getting better}
\end{Verbatim}


    \subsubsection{Define r\_h}\label{define-r_h}

R\_H is a factor that changes the population's growth rate

    \begin{Verbatim}[commandchars=\\\{\}]
{\color{incolor}In [{\color{incolor}3}]:} \PY{n}{R\PYZus{}H} \PY{o}{=} \PY{n}{random}\PY{o}{.}\PY{n}{uniform}\PY{p}{(}\PY{l+m+mi}{1}\PY{p}{,} \PY{l+m+mi}{5}\PY{p}{)}
        \PY{n}{R\PYZus{}H}
\end{Verbatim}


\begin{Verbatim}[commandchars=\\\{\}]
{\color{outcolor}Out[{\color{outcolor}3}]:} 3.485183155103463
\end{Verbatim}
            
    \subsubsection{Define P\_S}\label{define-p_s}

P\_S is a constant which defines the total size of the population on
which the simulation will be run.

    \begin{Verbatim}[commandchars=\\\{\}]
{\color{incolor}In [{\color{incolor}4}]:} \PY{n}{P\PYZus{}S} \PY{o}{=} \PY{l+m+mi}{300}
        \PY{n}{P\PYZus{}S}
\end{Verbatim}


\begin{Verbatim}[commandchars=\\\{\}]
{\color{outcolor}Out[{\color{outcolor}4}]:} 300
\end{Verbatim}
            
    \subsubsection{Set C\_H}\label{set-c_h}

    \begin{Verbatim}[commandchars=\\\{\}]
{\color{incolor}In [{\color{incolor}5}]:} \PY{n}{C\PYZus{}H} \PY{o}{=} \PY{n}{random}\PY{o}{.}\PY{n}{randint}\PY{p}{(}\PY{n}{P\PYZus{}S}\PY{p}{,} \PY{n}{P\PYZus{}S} \PY{o}{+} \PY{l+m+mi}{100}\PY{p}{)}
        \PY{n}{C\PYZus{}H}
\end{Verbatim}


\begin{Verbatim}[commandchars=\\\{\}]
{\color{outcolor}Out[{\color{outcolor}5}]:} 324
\end{Verbatim}
            
    \subsubsection{Define population growth
functions}\label{define-population-growth-functions}

For each of the host we define a different growth function to analize a
different behaviour for each of the virus strains

\begin{verbatim}
To make the work easier an exponential calculus function which allowed us to retrieve the probability value of MU, RO, and GAMMA whenever needed.

A recurent function was defined for NH that is used to calculate the growth rate which needs its previous time. This time is initialy Nh(0).

Now that we have Nh and exponential of MU we can easiy calculate the growth speed for each of the strains on a time basis.
\end{verbatim}

    \begin{Verbatim}[commandchars=\\\{\}]
{\color{incolor}In [{\color{incolor}6}]:} \PY{k+kn}{import} \PY{n+nn}{math}
        
        \PY{k}{def} \PY{n+nf}{get\PYZus{}exp\PYZus{}m}\PY{p}{(}\PY{n}{m}\PY{p}{,} \PY{n}{host\PYZus{}index}\PY{p}{,} \PY{n}{virus\PYZus{}index}\PY{p}{)}\PY{p}{:}
            \PY{l+s+sd}{\PYZdq{}\PYZdq{}\PYZdq{}}
        \PY{l+s+sd}{    Helper function for computing the exponential of}
        \PY{l+s+sd}{    probability value}
        \PY{l+s+sd}{    PARAMS:}
        \PY{l+s+sd}{    m: type of matrix to compute the exponentiation}
        \PY{l+s+sd}{    host\PYZus{}index: index of the host group}
        \PY{l+s+sd}{    virus\PYZus{}index: index of the virus strain}
        \PY{l+s+sd}{    RETURNS:}
        \PY{l+s+sd}{    exponentiation of the probability}
        \PY{l+s+sd}{    \PYZdq{}\PYZdq{}\PYZdq{}}
            \PY{k}{return} \PY{n}{math}\PY{o}{.}\PY{n}{exp}\PY{p}{(}\PY{o}{\PYZhy{}}\PY{n}{m}\PY{p}{[}\PY{n}{host\PYZus{}index}\PY{p}{]}\PY{p}{[}\PY{n}{virus\PYZus{}index}\PY{p}{]}\PY{p}{)}
        
        \PY{k}{def} \PY{n+nf}{get\PYZus{}nh}\PY{p}{(}\PY{n}{h\PYZus{}index}\PY{p}{,} \PY{n}{t}\PY{p}{)}\PY{p}{:}
            \PY{l+s+sd}{\PYZdq{}\PYZdq{}\PYZdq{}}
        \PY{l+s+sd}{    Computes the Nh factor}
        \PY{l+s+sd}{    PARAMS:}
        \PY{l+s+sd}{    h\PYZus{}index: index of the host group}
        \PY{l+s+sd}{    t: current timestep}
        \PY{l+s+sd}{    RETURNS:}
        \PY{l+s+sd}{    Nh factor}
        \PY{l+s+sd}{    \PYZdq{}\PYZdq{}\PYZdq{}}
            \PY{k}{def} \PY{n+nf}{get\PYZus{}nh\PYZus{}recur}\PY{p}{(}\PY{n}{current\PYZus{}t}\PY{p}{)}\PY{p}{:}
                \PY{k}{if} \PY{n}{current\PYZus{}t} \PY{o}{==} \PY{l+m+mi}{0}\PY{p}{:}
                    \PY{k}{return} \PY{n}{P\PYZus{}S}
                \PY{k}{else}\PY{p}{:}
                    \PY{n}{nh\PYZus{}past\PYZus{}times\PYZus{}exp} \PY{o}{=} \PY{n}{math}\PY{o}{.}\PY{n}{exp}\PY{p}{(}\PY{o}{\PYZhy{}}\PY{n}{MU}\PY{p}{[}\PY{n}{h\PYZus{}index}\PY{p}{]}\PY{p}{)} \PY{o}{*} \PY{n}{get\PYZus{}nh\PYZus{}recur}\PY{p}{(}\PY{n}{current\PYZus{}t} \PY{o}{\PYZhy{}} \PY{l+m+mi}{1}\PY{p}{)}
                    \PY{k}{return} \PY{n}{R\PYZus{}H} \PY{o}{*} \PY{n}{nh\PYZus{}past\PYZus{}times\PYZus{}exp} \PY{o}{*} \PY{p}{(}\PY{l+m+mi}{1} \PY{o}{\PYZhy{}} \PY{p}{(}\PY{n}{nh\PYZus{}past\PYZus{}times\PYZus{}exp} \PY{o}{/} \PY{n}{C\PYZus{}H}\PY{p}{)}\PY{p}{)} \PY{o}{+} \PY{n}{nh\PYZus{}past\PYZus{}times\PYZus{}exp}
        
            \PY{k}{if} \PY{n}{h\PYZus{}index} \PY{o}{==} \PY{l+m+mi}{0}\PY{p}{:}
                \PY{k}{return} \PY{n}{get\PYZus{}nh\PYZus{}recur}\PY{p}{(}\PY{n}{t}\PY{p}{)}
            \PY{k}{else}\PY{p}{:}
                \PY{k}{return} \PY{n}{P\PYZus{}S}
        
        \PY{k}{def} \PY{n+nf}{get\PYZus{}gh}\PY{p}{(}\PY{n}{h\PYZus{}i}\PY{p}{,} \PY{n}{t}\PY{p}{)}\PY{p}{:}
            \PY{l+s+sd}{\PYZdq{}\PYZdq{}\PYZdq{}}
        \PY{l+s+sd}{    Computes the Gh factor}
        \PY{l+s+sd}{    PARAMS:}
        \PY{l+s+sd}{    h\PYZus{}i: index of the host group}
        \PY{l+s+sd}{    t: current timestep}
        \PY{l+s+sd}{    RETURNS:}
        \PY{l+s+sd}{    Gh factor}
        \PY{l+s+sd}{    \PYZdq{}\PYZdq{}\PYZdq{}}    
            \PY{n}{nh} \PY{o}{=} \PY{n}{get\PYZus{}nh}\PY{p}{(}\PY{n}{h\PYZus{}i}\PY{p}{,} \PY{n}{t}\PY{p}{)}
            \PY{k}{if} \PY{n}{h\PYZus{}i} \PY{o}{==} \PY{l+m+mi}{0}\PY{p}{:}
                \PY{k}{return} \PY{n}{R\PYZus{}H} \PY{o}{*} \PY{n}{math}\PY{o}{.}\PY{n}{exp}\PY{p}{(}\PY{o}{\PYZhy{}}\PY{n}{MU}\PY{p}{[}\PY{n}{h\PYZus{}i}\PY{p}{]}\PY{p}{)} \PY{o}{*} \PY{n}{nh} \PY{o}{*} \PY{p}{(}\PY{l+m+mi}{1} \PY{o}{\PYZhy{}} \PY{p}{(}\PY{n}{math}\PY{o}{.}\PY{n}{exp}\PY{p}{(}\PY{o}{\PYZhy{}}\PY{n}{MU}\PY{p}{[}\PY{n}{h\PYZus{}i}\PY{p}{]}\PY{p}{)} \PY{o}{*} \PY{n}{nh}\PY{p}{)}\PY{o}{/}\PY{n}{C\PYZus{}H}\PY{p}{)}
            \PY{k}{else}\PY{p}{:}
                \PY{k}{return} \PY{p}{(}\PY{l+m+mi}{1} \PY{o}{\PYZhy{}} \PY{n}{math}\PY{o}{.}\PY{n}{exp}\PY{p}{(}\PY{o}{\PYZhy{}}\PY{n}{MU}\PY{p}{[}\PY{n}{h\PYZus{}i}\PY{p}{]}\PY{p}{)}\PY{p}{)} \PY{o}{*} \PY{n}{nh}
                
            
\end{Verbatim}


    \begin{Verbatim}[commandchars=\\\{\}]
{\color{incolor}In [{\color{incolor}7}]:} \PY{n}{PAST\PYZus{}R} \PY{o}{=} \PY{p}{[}\PY{p}{[}\PY{p}{\PYZob{}}\PY{p}{\PYZcb{}} \PY{k}{for} \PY{n}{\PYZus{}} \PY{o+ow}{in} \PY{n+nb}{range}\PY{p}{(}\PY{n}{N\PYZus{}V}\PY{p}{)}\PY{p}{]} \PY{k}{for} \PY{n}{\PYZus{}} \PY{o+ow}{in} \PY{n+nb}{range}\PY{p}{(}\PY{n}{N\PYZus{}H}\PY{p}{)}\PY{p}{]}
        \PY{n}{PAST\PYZus{}I} \PY{o}{=} \PY{p}{[}\PY{p}{[}\PY{p}{\PYZob{}}\PY{p}{\PYZcb{}} \PY{k}{for} \PY{n}{\PYZus{}} \PY{o+ow}{in} \PY{n+nb}{range}\PY{p}{(}\PY{n}{N\PYZus{}V}\PY{p}{)}\PY{p}{]} \PY{k}{for} \PY{n}{\PYZus{}} \PY{o+ow}{in} \PY{n+nb}{range}\PY{p}{(}\PY{n}{N\PYZus{}H}\PY{p}{)}\PY{p}{]}
        \PY{n}{PAST\PYZus{}S} \PY{o}{=} \PY{p}{[}\PY{p}{\PYZob{}}\PY{p}{\PYZcb{}} \PY{k}{for} \PY{n}{\PYZus{}} \PY{o+ow}{in} \PY{n+nb}{range}\PY{p}{(}\PY{n}{N\PYZus{}H}\PY{p}{)}\PY{p}{]}
        
        
        \PY{k}{def} \PY{n+nf}{get\PYZus{}s}\PY{p}{(}\PY{n}{h\PYZus{}index}\PY{p}{,} \PY{n}{t}\PY{p}{)}\PY{p}{:}
            \PY{l+s+sd}{\PYZdq{}\PYZdq{}\PYZdq{}}
        \PY{l+s+sd}{    Computes the Succeptibility}
        \PY{l+s+sd}{    PARAMS:}
        \PY{l+s+sd}{    h\PYZus{}iindex: index of the host group}
        \PY{l+s+sd}{    t: current timestep}
        \PY{l+s+sd}{    RETURNS:}
        \PY{l+s+sd}{    Secceptibility}
        \PY{l+s+sd}{    \PYZdq{}\PYZdq{}\PYZdq{}}  
            \PY{n}{ans} \PY{o}{=} \PY{n}{PAST\PYZus{}S}\PY{p}{[}\PY{n}{h\PYZus{}index}\PY{p}{]}\PY{o}{.}\PY{n}{get}\PY{p}{(}\PY{n}{t}\PY{p}{,} \PY{k+kc}{None}\PY{p}{)}
            \PY{k}{if} \PY{n}{ans} \PY{o+ow}{is} \PY{o+ow}{not} \PY{k+kc}{None}\PY{p}{:}
                \PY{k}{return} \PY{n}{ans}
            \PY{c+c1}{\PYZsh{}Stop condition}
            \PY{k}{if} \PY{n}{t} \PY{o}{==} \PY{l+m+mi}{0}\PY{p}{:}
                \PY{n}{PAST\PYZus{}S}\PY{p}{[}\PY{n}{h\PYZus{}index}\PY{p}{]}\PY{p}{[}\PY{n}{t}\PY{p}{]} \PY{o}{=} \PY{l+m+mi}{100}
                \PY{k}{return} \PY{n}{PAST\PYZus{}S}\PY{p}{[}\PY{n}{h\PYZus{}index}\PY{p}{]}\PY{p}{[}\PY{n}{t}\PY{p}{]}
            \PY{k}{else}\PY{p}{:}
                \PY{n}{past} \PY{o}{=} \PY{n}{PAST\PYZus{}S}\PY{p}{[}\PY{n}{h\PYZus{}index}\PY{p}{]}\PY{o}{.}\PY{n}{get}\PY{p}{(}\PY{n}{t}\PY{o}{\PYZhy{}}\PY{l+m+mi}{1}\PY{p}{,} \PY{k+kc}{None}\PY{p}{)}
                \PY{k}{if} \PY{n}{past} \PY{o}{==} \PY{k+kc}{None}\PY{p}{:}
                    \PY{n}{PAST\PYZus{}S}\PY{p}{[}\PY{n}{h\PYZus{}index}\PY{p}{]}\PY{p}{[}\PY{n}{t}\PY{o}{\PYZhy{}}\PY{l+m+mi}{1}\PY{p}{]} \PY{o}{=} \PY{n}{get\PYZus{}s}\PY{p}{(}\PY{n}{h\PYZus{}index}\PY{p}{,} \PY{n}{t}\PY{o}{\PYZhy{}}\PY{l+m+mi}{1}\PY{p}{)}
                    \PY{n}{past} \PY{o}{=} \PY{n}{PAST\PYZus{}S}\PY{p}{[}\PY{n}{h\PYZus{}index}\PY{p}{]}\PY{p}{[}\PY{n}{t}\PY{o}{\PYZhy{}}\PY{l+m+mi}{1}\PY{p}{]}
                \PY{n}{PAST\PYZus{}S}\PY{p}{[}\PY{n}{h\PYZus{}index}\PY{p}{]}\PY{p}{[}\PY{n}{t}\PY{p}{]} \PY{o}{=} \PY{n}{get\PYZus{}gh}\PY{p}{(}\PY{n}{h\PYZus{}index}\PY{p}{,} \PY{n}{t}\PY{o}{\PYZhy{}}\PY{l+m+mi}{1}\PY{p}{)} \PY{o}{+} \PY{n}{math}\PY{o}{.}\PY{n}{exp}\PY{p}{(}\PY{o}{\PYZhy{}}\PY{n}{MU}\PY{p}{[}\PY{n}{h\PYZus{}index}\PY{p}{]}\PY{p}{)} \PY{o}{*} \PY{n}{math}\PY{o}{.}\PY{n}{exp}\PY{p}{(}\PY{o}{\PYZhy{}}\PY{n+nb}{sum}\PY{p}{(}\PY{n}{BETA}\PY{p}{[}\PY{n}{h\PYZus{}index}\PY{p}{]}\PY{p}{[}\PY{n}{v\PYZus{}index}\PY{p}{]} \PY{o}{*} \PY{n}{get\PYZus{}i}\PY{p}{(}\PY{n}{h\PYZus{}index}\PY{p}{,} \PY{n}{v\PYZus{}index}\PY{p}{,} \PY{n}{t}\PY{o}{\PYZhy{}}\PY{l+m+mi}{1}\PY{p}{)} \PY{k}{for} \PY{n}{v\PYZus{}index} \PY{o+ow}{in} \PY{n+nb}{range}\PY{p}{(}\PY{n}{N\PYZus{}V}\PY{p}{)}\PY{p}{)}\PY{p}{)} \PY{o}{*} \PY{n}{past} \PY{o}{+} \PY{n+nb}{sum}\PY{p}{(}\PY{n}{get\PYZus{}r}\PY{p}{(}\PY{n}{h\PYZus{}index}\PY{p}{,} \PY{n}{v\PYZus{}index}\PY{p}{,} \PY{n}{t}\PY{o}{\PYZhy{}}\PY{l+m+mi}{1}\PY{p}{)} \PY{o}{*} \PY{n}{math}\PY{o}{.}\PY{n}{exp}\PY{p}{(}\PY{o}{\PYZhy{}}\PY{n}{MU}\PY{p}{[}\PY{n}{h\PYZus{}index}\PY{p}{]}\PY{p}{)} \PY{o}{*} \PY{p}{(}\PY{l+m+mi}{1} \PY{o}{\PYZhy{}} \PY{n}{get\PYZus{}exp\PYZus{}m}\PY{p}{(}\PY{n}{RO}\PY{p}{,} \PY{n}{h\PYZus{}index}\PY{p}{,} \PY{n}{v\PYZus{}index}\PY{p}{)}\PY{p}{)} \PY{k}{for} \PY{n}{v\PYZus{}index} \PY{o+ow}{in} \PY{n+nb}{range}\PY{p}{(}\PY{n}{N\PYZus{}V}\PY{p}{)}\PY{p}{)}
                \PY{k}{return} \PY{n}{PAST\PYZus{}S}\PY{p}{[}\PY{n}{h\PYZus{}index}\PY{p}{]}\PY{p}{[}\PY{n}{t}\PY{p}{]}
        
        \PY{k}{def} \PY{n+nf}{get\PYZus{}i}\PY{p}{(}\PY{n}{h\PYZus{}i}\PY{p}{,} \PY{n}{v\PYZus{}i}\PY{p}{,} \PY{n}{t}\PY{p}{)}\PY{p}{:}
            \PY{n}{ans} \PY{o}{=} \PY{n}{PAST\PYZus{}I}\PY{p}{[}\PY{n}{h\PYZus{}i}\PY{p}{]}\PY{p}{[}\PY{n}{v\PYZus{}i}\PY{p}{]}\PY{o}{.}\PY{n}{get}\PY{p}{(}\PY{n}{t}\PY{p}{,} \PY{k+kc}{None}\PY{p}{)}
            \PY{k}{if} \PY{n}{ans} \PY{o+ow}{is} \PY{o+ow}{not} \PY{k+kc}{None}\PY{p}{:}
                \PY{k}{return} \PY{n}{ans}
            \PY{k}{if} \PY{n}{t} \PY{o}{==} \PY{l+m+mi}{0}\PY{p}{:}
                \PY{n}{PAST\PYZus{}I}\PY{p}{[}\PY{n}{h\PYZus{}i}\PY{p}{]}\PY{p}{[}\PY{n}{v\PYZus{}i}\PY{p}{]}\PY{p}{[}\PY{n}{t}\PY{p}{]} \PY{o}{=} \PY{l+m+mi}{100}
                \PY{k}{return} \PY{n}{PAST\PYZus{}I}\PY{p}{[}\PY{n}{h\PYZus{}i}\PY{p}{]}\PY{p}{[}\PY{n}{v\PYZus{}i}\PY{p}{]}\PY{p}{[}\PY{n}{t}\PY{p}{]}
            \PY{k}{else}\PY{p}{:}
                \PY{n}{past\PYZus{}s} \PY{o}{=} \PY{n}{PAST\PYZus{}S}\PY{p}{[}\PY{n}{h\PYZus{}i}\PY{p}{]}\PY{o}{.}\PY{n}{get}\PY{p}{(}\PY{n}{t} \PY{o}{\PYZhy{}} \PY{l+m+mi}{1}\PY{p}{,} \PY{k+kc}{None}\PY{p}{)}
                \PY{k}{if} \PY{n}{past\PYZus{}s} \PY{o+ow}{is} \PY{k+kc}{None}\PY{p}{:}
                    \PY{n}{past\PYZus{}s} \PY{o}{=} \PY{n}{get\PYZus{}s}\PY{p}{(}\PY{n}{h\PYZus{}i}\PY{p}{,} \PY{n}{t} \PY{o}{\PYZhy{}} \PY{l+m+mi}{1}\PY{p}{)}
                \PY{n}{past\PYZus{}i} \PY{o}{=} \PY{n}{PAST\PYZus{}I}\PY{p}{[}\PY{n}{h\PYZus{}i}\PY{p}{]}\PY{p}{[}\PY{n}{v\PYZus{}i}\PY{p}{]}\PY{o}{.}\PY{n}{get}\PY{p}{(}\PY{n}{t} \PY{o}{\PYZhy{}} \PY{l+m+mi}{1}\PY{p}{,} \PY{k+kc}{None}\PY{p}{)}
                \PY{k}{if} \PY{n}{past\PYZus{}i} \PY{o+ow}{is} \PY{k+kc}{None}\PY{p}{:}
                    \PY{n}{past\PYZus{}i} \PY{o}{=} \PY{n}{get\PYZus{}i}\PY{p}{(}\PY{n}{h\PYZus{}i}\PY{p}{,} \PY{n}{v\PYZus{}i}\PY{p}{,} \PY{n}{t} \PY{o}{\PYZhy{}} \PY{l+m+mi}{1}\PY{p}{)}
                    \PY{n}{PAST\PYZus{}I}\PY{p}{[}\PY{n}{h\PYZus{}i}\PY{p}{]}\PY{p}{[}\PY{n}{v\PYZus{}i}\PY{p}{]}\PY{p}{[}\PY{n}{t} \PY{o}{\PYZhy{}} \PY{l+m+mi}{1}\PY{p}{]} \PY{o}{=} \PY{n}{past\PYZus{}i}
                \PY{n}{PAST\PYZus{}I}\PY{p}{[}\PY{n}{h\PYZus{}i}\PY{p}{]}\PY{p}{[}\PY{n}{v\PYZus{}i}\PY{p}{]}\PY{p}{[}\PY{n}{t}\PY{p}{]} \PY{o}{=} \PY{n}{past\PYZus{}s} \PY{o}{*} \PY{n}{math}\PY{o}{.}\PY{n}{exp}\PY{p}{(}\PY{o}{\PYZhy{}}\PY{n}{MU}\PY{p}{[}\PY{n}{h\PYZus{}i}\PY{p}{]}\PY{p}{)} \PY{o}{*} \PY{p}{(}\PY{l+m+mi}{1} \PY{o}{\PYZhy{}} \PY{n}{math}\PY{o}{.}\PY{n}{exp}\PY{p}{(}\PY{o}{\PYZhy{}}\PY{n+nb}{sum}\PY{p}{(}\PY{n}{BETA}\PY{p}{[}\PY{n}{h\PYZus{}i}\PY{p}{]}\PY{p}{[}\PY{n}{v\PYZus{}index}\PY{p}{]} \PY{o}{*} \PY{n}{get\PYZus{}i}\PY{p}{(}\PY{n}{h\PYZus{}i}\PY{p}{,} \PY{n}{v\PYZus{}index}\PY{p}{,} \PY{n}{t}\PY{o}{\PYZhy{}}\PY{l+m+mi}{1}\PY{p}{)} \PY{k}{for} \PY{n}{v\PYZus{}index} \PY{o+ow}{in} \PY{n+nb}{range}\PY{p}{(}\PY{n}{N\PYZus{}V}\PY{p}{)}\PY{p}{)}\PY{p}{)}\PY{p}{)} \PY{o}{*} \PY{p}{(}\PY{p}{(}\PY{n}{BETA}\PY{p}{[}\PY{n}{h\PYZus{}i}\PY{p}{]}\PY{p}{[}\PY{n}{v\PYZus{}i}\PY{p}{]} \PY{o}{*} \PY{n}{past\PYZus{}i}\PY{p}{)}\PY{o}{/}\PY{p}{(}\PY{n+nb}{sum}\PY{p}{(}\PY{n}{BETA}\PY{p}{[}\PY{n}{h\PYZus{}i}\PY{p}{]}\PY{p}{[}\PY{n}{v\PYZus{}index}\PY{p}{]} \PY{o}{*} \PY{n}{get\PYZus{}i}\PY{p}{(}\PY{n}{h\PYZus{}i}\PY{p}{,} \PY{n}{v\PYZus{}index}\PY{p}{,} \PY{n}{t}\PY{o}{\PYZhy{}}\PY{l+m+mi}{1}\PY{p}{)} \PY{k}{for} \PY{n}{v\PYZus{}index} \PY{o+ow}{in} \PY{n+nb}{range}\PY{p}{(}\PY{n}{N\PYZus{}V}\PY{p}{)}\PY{p}{)}\PY{p}{)}\PY{p}{)} \PY{o}{+} \PY{n}{past\PYZus{}i} \PY{o}{*} \PY{n}{math}\PY{o}{.}\PY{n}{exp}\PY{p}{(}\PY{o}{\PYZhy{}}\PY{n}{MU}\PY{p}{[}\PY{n}{h\PYZus{}i}\PY{p}{]}\PY{p}{)} \PY{o}{*} \PY{n}{get\PYZus{}exp\PYZus{}m}\PY{p}{(}\PY{n}{GAMMA}\PY{p}{,} \PY{n}{h\PYZus{}i}\PY{p}{,} \PY{n}{v\PYZus{}i}\PY{p}{)}
                \PY{k}{return} \PY{n}{PAST\PYZus{}I}\PY{p}{[}\PY{n}{h\PYZus{}i}\PY{p}{]}\PY{p}{[}\PY{n}{v\PYZus{}i}\PY{p}{]}\PY{p}{[}\PY{n}{t}\PY{p}{]}
        
        
        \PY{k}{def} \PY{n+nf}{get\PYZus{}r}\PY{p}{(}\PY{n}{h\PYZus{}i}\PY{p}{,} \PY{n}{v\PYZus{}i}\PY{p}{,} \PY{n}{t}\PY{p}{)}\PY{p}{:}
            \PY{n}{ans} \PY{o}{=} \PY{n}{PAST\PYZus{}R}\PY{p}{[}\PY{n}{h\PYZus{}i}\PY{p}{]}\PY{p}{[}\PY{n}{v\PYZus{}i}\PY{p}{]}\PY{o}{.}\PY{n}{get}\PY{p}{(}\PY{n}{t}\PY{p}{,} \PY{k+kc}{None}\PY{p}{)}
            \PY{k}{if} \PY{n}{ans} \PY{o+ow}{is} \PY{o+ow}{not} \PY{k+kc}{None}\PY{p}{:}
                \PY{k}{return} \PY{n}{ans}
            \PY{k}{if} \PY{n}{t} \PY{o}{==} \PY{l+m+mi}{0}\PY{p}{:}
                \PY{n}{PAST\PYZus{}R}\PY{p}{[}\PY{n}{h\PYZus{}i}\PY{p}{]}\PY{p}{[}\PY{n}{v\PYZus{}i}\PY{p}{]}\PY{p}{[}\PY{n}{t}\PY{p}{]} \PY{o}{=} \PY{l+m+mi}{100}
                \PY{k}{return} \PY{n}{PAST\PYZus{}R}\PY{p}{[}\PY{n}{h\PYZus{}i}\PY{p}{]}\PY{p}{[}\PY{n}{v\PYZus{}i}\PY{p}{]}\PY{p}{[}\PY{n}{t}\PY{p}{]}
            \PY{k}{else}\PY{p}{:}
                \PY{n}{past\PYZus{}r} \PY{o}{=} \PY{n}{PAST\PYZus{}R}\PY{p}{[}\PY{n}{h\PYZus{}i}\PY{p}{]}\PY{p}{[}\PY{n}{v\PYZus{}i}\PY{p}{]}\PY{o}{.}\PY{n}{get}\PY{p}{(}\PY{n}{t}\PY{o}{\PYZhy{}}\PY{l+m+mi}{1}\PY{p}{,} \PY{k+kc}{None}\PY{p}{)}
                \PY{k}{if} \PY{n}{past\PYZus{}r} \PY{o+ow}{is} \PY{k+kc}{None}\PY{p}{:}
                    \PY{n}{PAST\PYZus{}R}\PY{p}{[}\PY{n}{h\PYZus{}i}\PY{p}{]}\PY{p}{[}\PY{n}{v\PYZus{}i}\PY{p}{]}\PY{p}{[}\PY{n}{t}\PY{o}{\PYZhy{}}\PY{l+m+mi}{1}\PY{p}{]} \PY{o}{=} \PY{n}{get\PYZus{}r}\PY{p}{(}\PY{n}{h\PYZus{}i}\PY{p}{,} \PY{n}{v\PYZus{}i}\PY{p}{,} \PY{n}{t}\PY{o}{\PYZhy{}}\PY{l+m+mi}{1}\PY{p}{)}
                    \PY{n}{past\PYZus{}r} \PY{o}{=} \PY{n}{PAST\PYZus{}R}\PY{p}{[}\PY{n}{h\PYZus{}i}\PY{p}{]}\PY{p}{[}\PY{n}{v\PYZus{}i}\PY{p}{]}\PY{p}{[}\PY{n}{t}\PY{o}{\PYZhy{}}\PY{l+m+mi}{1}\PY{p}{]}
                \PY{n}{past\PYZus{}i} \PY{o}{=} \PY{n}{PAST\PYZus{}I}\PY{p}{[}\PY{n}{h\PYZus{}i}\PY{p}{]}\PY{p}{[}\PY{n}{v\PYZus{}i}\PY{p}{]}\PY{o}{.}\PY{n}{get}\PY{p}{(}\PY{n}{t}\PY{o}{\PYZhy{}}\PY{l+m+mi}{1}\PY{p}{,} \PY{k+kc}{None}\PY{p}{)}
                \PY{k}{if} \PY{n}{past\PYZus{}i} \PY{o+ow}{is} \PY{k+kc}{None}\PY{p}{:}
                    \PY{n}{past\PYZus{}i} \PY{o}{=} \PY{n}{get\PYZus{}i}\PY{p}{(}\PY{n}{h\PYZus{}i}\PY{p}{,} \PY{n}{v\PYZus{}i}\PY{p}{,} \PY{n}{t}\PY{o}{\PYZhy{}}\PY{l+m+mi}{1}\PY{p}{)}
                \PY{n}{PAST\PYZus{}R}\PY{p}{[}\PY{n}{h\PYZus{}i}\PY{p}{]}\PY{p}{[}\PY{n}{v\PYZus{}i}\PY{p}{]}\PY{p}{[}\PY{n}{t}\PY{p}{]} \PY{o}{=} \PY{n}{past\PYZus{}r} \PY{o}{*} \PY{n}{math}\PY{o}{.}\PY{n}{exp}\PY{p}{(}\PY{o}{\PYZhy{}}\PY{n}{MU}\PY{p}{[}\PY{n}{h\PYZus{}i}\PY{p}{]}\PY{p}{)} \PY{o}{*} \PY{n}{get\PYZus{}exp\PYZus{}m}\PY{p}{(}\PY{n}{RO}\PY{p}{,} \PY{n}{h\PYZus{}i}\PY{p}{,} \PY{n}{v\PYZus{}i}\PY{p}{)} \PY{o}{+} \PY{n}{past\PYZus{}i} \PY{o}{*} \PY{n}{math}\PY{o}{.}\PY{n}{exp}\PY{p}{(}\PY{o}{\PYZhy{}}\PY{n}{MU}\PY{p}{[}\PY{n}{h\PYZus{}i}\PY{p}{]}\PY{p}{)} \PY{o}{*} \PY{p}{(}\PY{l+m+mi}{1} \PY{o}{\PYZhy{}} \PY{n}{get\PYZus{}exp\PYZus{}m}\PY{p}{(}\PY{n}{GAMMA}\PY{p}{,} \PY{n}{h\PYZus{}i}\PY{p}{,} \PY{n}{v\PYZus{}i}\PY{p}{)}\PY{p}{)}
                \PY{k}{return} \PY{n}{PAST\PYZus{}R}\PY{p}{[}\PY{n}{h\PYZus{}i}\PY{p}{]}\PY{p}{[}\PY{n}{v\PYZus{}i}\PY{p}{]}\PY{p}{[}\PY{n}{t}\PY{p}{]}
\end{Verbatim}


    \subsection{Graphs}\label{graphs}

\#\#\#\# 4 diferent tests were graphed, all of these use "Time" and
"Population size" in order to demonstrate the results of the simulation.
The following cases were tested: \emph{Adults with agressive virus
strain (Ebola) }Children with agresive virus strain(Ebola) \emph{Adults
with soft virus strain(Flu) }Children with agresive virus strain(Flu)

    \begin{Verbatim}[commandchars=\\\{\}]
{\color{incolor}In [{\color{incolor}8}]:} \PY{k+kn}{import} \PY{n+nn}{matplotlib}\PY{n+nn}{.}\PY{n+nn}{pyplot} \PY{k}{as} \PY{n+nn}{plt}
        \PY{k+kn}{import} \PY{n+nn}{matplotlib}
\end{Verbatim}


    \begin{Verbatim}[commandchars=\\\{\}]
{\color{incolor}In [{\color{incolor}9}]:} \PY{n}{matplotlib}\PY{o}{.}\PY{n}{rcParams}\PY{p}{[}\PY{l+s+s1}{\PYZsq{}}\PY{l+s+s1}{figure.figsize}\PY{l+s+s1}{\PYZsq{}}\PY{p}{]} \PY{o}{=} \PY{p}{[}\PY{l+m+mi}{15}\PY{p}{,} \PY{l+m+mi}{7}\PY{p}{]}
        \PY{n}{plt}\PY{o}{.}\PY{n}{xlabel}\PY{p}{(}\PY{l+s+s1}{\PYZsq{}}\PY{l+s+s1}{Timestep}\PY{l+s+s1}{\PYZsq{}}\PY{p}{)}
        \PY{n}{plt}\PY{o}{.}\PY{n}{ylabel}\PY{p}{(}\PY{l+s+s1}{\PYZsq{}}\PY{l+s+s1}{Population size}\PY{l+s+s1}{\PYZsq{}}\PY{p}{)}
        \PY{n}{plt}\PY{o}{.}\PY{n}{title}\PY{p}{(}\PY{l+s+s1}{\PYZsq{}}\PY{l+s+s1}{Adults with aggressive virus strain}\PY{l+s+s1}{\PYZsq{}}\PY{p}{)}
        \PY{n}{plt}\PY{o}{.}\PY{n}{plot}\PY{p}{(}\PY{p}{[}\PY{n}{get\PYZus{}s}\PY{p}{(}\PY{l+m+mi}{0}\PY{p}{,} \PY{n}{x}\PY{p}{)} \PY{k}{for} \PY{n}{x} \PY{o+ow}{in} \PY{n+nb}{range}\PY{p}{(}\PY{l+m+mi}{100}\PY{p}{)}\PY{p}{]}\PY{p}{,} \PY{n}{label}\PY{o}{=}\PY{l+s+s1}{\PYZsq{}}\PY{l+s+s1}{Susceptible}\PY{l+s+s1}{\PYZsq{}}\PY{p}{)}
        \PY{n}{plt}\PY{o}{.}\PY{n}{plot}\PY{p}{(}\PY{p}{[}\PY{n}{get\PYZus{}i}\PY{p}{(}\PY{l+m+mi}{0}\PY{p}{,} \PY{l+m+mi}{0}\PY{p}{,} \PY{n}{x}\PY{p}{)} \PY{k}{for} \PY{n}{x} \PY{o+ow}{in} \PY{n+nb}{range}\PY{p}{(}\PY{l+m+mi}{100}\PY{p}{)}\PY{p}{]}\PY{p}{,} \PY{n}{label}\PY{o}{=}\PY{l+s+s1}{\PYZsq{}}\PY{l+s+s1}{Infected}\PY{l+s+s1}{\PYZsq{}}\PY{p}{)}
        \PY{n}{plt}\PY{o}{.}\PY{n}{plot}\PY{p}{(}\PY{p}{[}\PY{n}{get\PYZus{}r}\PY{p}{(}\PY{l+m+mi}{0}\PY{p}{,} \PY{l+m+mi}{0}\PY{p}{,} \PY{n}{x}\PY{p}{)} \PY{k}{for} \PY{n}{x} \PY{o+ow}{in} \PY{n+nb}{range}\PY{p}{(}\PY{l+m+mi}{100}\PY{p}{)}\PY{p}{]}\PY{p}{,} \PY{n}{label}\PY{o}{=}\PY{l+s+s1}{\PYZsq{}}\PY{l+s+s1}{Cured}\PY{l+s+s1}{\PYZsq{}}\PY{p}{)}
        \PY{n}{plt}\PY{o}{.}\PY{n}{legend}\PY{p}{(}\PY{p}{)}
\end{Verbatim}


\begin{Verbatim}[commandchars=\\\{\}]
{\color{outcolor}Out[{\color{outcolor}9}]:} <matplotlib.legend.Legend at 0x11106a978>
\end{Verbatim}
            
    \begin{center}
    \adjustimage{max size={0.9\linewidth}{0.9\paperheight}}{output_17_1.png}
    \end{center}
    { \hspace*{\fill} \\}
    
    \paragraph{As can be noted, the ebola virus has a high succeptibility
rate for older people, but only arround half of them get infected and
die. On the other hand the cured population drops almost to cero and
remains very low for this
case.}\label{as-can-be-noted-the-ebola-virus-has-a-high-succeptibility-rate-for-older-people-but-only-arround-half-of-them-get-infected-and-die.-on-the-other-hand-the-cured-population-drops-almost-to-cero-and-remains-very-low-for-this-case.}

    \begin{Verbatim}[commandchars=\\\{\}]
{\color{incolor}In [{\color{incolor}10}]:} \PY{n}{matplotlib}\PY{o}{.}\PY{n}{rcParams}\PY{p}{[}\PY{l+s+s1}{\PYZsq{}}\PY{l+s+s1}{figure.figsize}\PY{l+s+s1}{\PYZsq{}}\PY{p}{]} \PY{o}{=} \PY{p}{[}\PY{l+m+mi}{15}\PY{p}{,} \PY{l+m+mi}{7}\PY{p}{]}
         \PY{n}{plt}\PY{o}{.}\PY{n}{xlabel}\PY{p}{(}\PY{l+s+s1}{\PYZsq{}}\PY{l+s+s1}{Timestep}\PY{l+s+s1}{\PYZsq{}}\PY{p}{)}
         \PY{n}{plt}\PY{o}{.}\PY{n}{ylabel}\PY{p}{(}\PY{l+s+s1}{\PYZsq{}}\PY{l+s+s1}{Population size}\PY{l+s+s1}{\PYZsq{}}\PY{p}{)}
         \PY{n}{plt}\PY{o}{.}\PY{n}{title}\PY{p}{(}\PY{l+s+s1}{\PYZsq{}}\PY{l+s+s1}{Kids with aggressive virus strain}\PY{l+s+s1}{\PYZsq{}}\PY{p}{)}
         \PY{n}{plt}\PY{o}{.}\PY{n}{plot}\PY{p}{(}\PY{p}{[}\PY{n}{get\PYZus{}s}\PY{p}{(}\PY{l+m+mi}{1}\PY{p}{,} \PY{n}{x}\PY{p}{)} \PY{k}{for} \PY{n}{x} \PY{o+ow}{in} \PY{n+nb}{range}\PY{p}{(}\PY{l+m+mi}{100}\PY{p}{)}\PY{p}{]}\PY{p}{,} \PY{n}{label}\PY{o}{=}\PY{l+s+s1}{\PYZsq{}}\PY{l+s+s1}{Susceptible}\PY{l+s+s1}{\PYZsq{}}\PY{p}{)}
         \PY{n}{plt}\PY{o}{.}\PY{n}{plot}\PY{p}{(}\PY{p}{[}\PY{n}{get\PYZus{}i}\PY{p}{(}\PY{l+m+mi}{1}\PY{p}{,} \PY{l+m+mi}{0}\PY{p}{,} \PY{n}{x}\PY{p}{)} \PY{k}{for} \PY{n}{x} \PY{o+ow}{in} \PY{n+nb}{range}\PY{p}{(}\PY{l+m+mi}{100}\PY{p}{)}\PY{p}{]}\PY{p}{,} \PY{n}{label}\PY{o}{=}\PY{l+s+s1}{\PYZsq{}}\PY{l+s+s1}{Infected}\PY{l+s+s1}{\PYZsq{}}\PY{p}{)}
         \PY{n}{plt}\PY{o}{.}\PY{n}{plot}\PY{p}{(}\PY{p}{[}\PY{n}{get\PYZus{}r}\PY{p}{(}\PY{l+m+mi}{1}\PY{p}{,} \PY{l+m+mi}{0}\PY{p}{,} \PY{n}{x}\PY{p}{)} \PY{k}{for} \PY{n}{x} \PY{o+ow}{in} \PY{n+nb}{range}\PY{p}{(}\PY{l+m+mi}{100}\PY{p}{)}\PY{p}{]}\PY{p}{,} \PY{n}{label}\PY{o}{=}\PY{l+s+s1}{\PYZsq{}}\PY{l+s+s1}{Cured}\PY{l+s+s1}{\PYZsq{}}\PY{p}{)}
         \PY{n}{plt}\PY{o}{.}\PY{n}{legend}\PY{p}{(}\PY{p}{)}
\end{Verbatim}


\begin{Verbatim}[commandchars=\\\{\}]
{\color{outcolor}Out[{\color{outcolor}10}]:} <matplotlib.legend.Legend at 0x111497fd0>
\end{Verbatim}
            
    \begin{center}
    \adjustimage{max size={0.9\linewidth}{0.9\paperheight}}{output_19_1.png}
    \end{center}
    { \hspace*{\fill} \\}
    
    \subparagraph{When testing a children population with the agressive
virus strain case we can notice how the high probability for children to
get infected and the high rate of mortality of the Strain lead to the
death in many cases decreasing the population by arround a 1/3, and
droping the cure and succeptible chances to almost
cero.}\label{when-testing-a-children-population-with-the-agressive-virus-strain-case-we-can-notice-how-the-high-probability-for-children-to-get-infected-and-the-high-rate-of-mortality-of-the-strain-lead-to-the-death-in-many-cases-decreasing-the-population-by-arround-a-13-and-droping-the-cure-and-succeptible-chances-to-almost-cero.}

    \begin{Verbatim}[commandchars=\\\{\}]
{\color{incolor}In [{\color{incolor}11}]:} \PY{n}{matplotlib}\PY{o}{.}\PY{n}{rcParams}\PY{p}{[}\PY{l+s+s1}{\PYZsq{}}\PY{l+s+s1}{figure.figsize}\PY{l+s+s1}{\PYZsq{}}\PY{p}{]} \PY{o}{=} \PY{p}{[}\PY{l+m+mi}{15}\PY{p}{,} \PY{l+m+mi}{7}\PY{p}{]}
         \PY{n}{plt}\PY{o}{.}\PY{n}{xlabel}\PY{p}{(}\PY{l+s+s1}{\PYZsq{}}\PY{l+s+s1}{Timestep}\PY{l+s+s1}{\PYZsq{}}\PY{p}{)}
         \PY{n}{plt}\PY{o}{.}\PY{n}{ylabel}\PY{p}{(}\PY{l+s+s1}{\PYZsq{}}\PY{l+s+s1}{Population size}\PY{l+s+s1}{\PYZsq{}}\PY{p}{)}
         \PY{n}{plt}\PY{o}{.}\PY{n}{title}\PY{p}{(}\PY{l+s+s1}{\PYZsq{}}\PY{l+s+s1}{Adults with soft virus strain}\PY{l+s+s1}{\PYZsq{}}\PY{p}{)}
         \PY{n}{plt}\PY{o}{.}\PY{n}{plot}\PY{p}{(}\PY{p}{[}\PY{n}{get\PYZus{}s}\PY{p}{(}\PY{l+m+mi}{0}\PY{p}{,} \PY{n}{x}\PY{p}{)} \PY{k}{for} \PY{n}{x} \PY{o+ow}{in} \PY{n+nb}{range}\PY{p}{(}\PY{l+m+mi}{100}\PY{p}{)}\PY{p}{]}\PY{p}{,} \PY{n}{label}\PY{o}{=}\PY{l+s+s1}{\PYZsq{}}\PY{l+s+s1}{Susceptible}\PY{l+s+s1}{\PYZsq{}}\PY{p}{)}
         \PY{n}{plt}\PY{o}{.}\PY{n}{plot}\PY{p}{(}\PY{p}{[}\PY{n}{get\PYZus{}i}\PY{p}{(}\PY{l+m+mi}{0}\PY{p}{,} \PY{l+m+mi}{1}\PY{p}{,} \PY{n}{x}\PY{p}{)} \PY{k}{for} \PY{n}{x} \PY{o+ow}{in} \PY{n+nb}{range}\PY{p}{(}\PY{l+m+mi}{100}\PY{p}{)}\PY{p}{]}\PY{p}{,} \PY{n}{label}\PY{o}{=}\PY{l+s+s1}{\PYZsq{}}\PY{l+s+s1}{Infected}\PY{l+s+s1}{\PYZsq{}}\PY{p}{)}
         \PY{n}{plt}\PY{o}{.}\PY{n}{plot}\PY{p}{(}\PY{p}{[}\PY{n}{get\PYZus{}r}\PY{p}{(}\PY{l+m+mi}{0}\PY{p}{,} \PY{l+m+mi}{1}\PY{p}{,} \PY{n}{x}\PY{p}{)} \PY{k}{for} \PY{n}{x} \PY{o+ow}{in} \PY{n+nb}{range}\PY{p}{(}\PY{l+m+mi}{100}\PY{p}{)}\PY{p}{]}\PY{p}{,} \PY{n}{label}\PY{o}{=}\PY{l+s+s1}{\PYZsq{}}\PY{l+s+s1}{Cured}\PY{l+s+s1}{\PYZsq{}}\PY{p}{)}
         \PY{n}{plt}\PY{o}{.}\PY{n}{legend}\PY{p}{(}\PY{p}{)}
\end{Verbatim}


\begin{Verbatim}[commandchars=\\\{\}]
{\color{outcolor}Out[{\color{outcolor}11}]:} <matplotlib.legend.Legend at 0x1114a6c50>
\end{Verbatim}
            
    \begin{center}
    \adjustimage{max size={0.9\linewidth}{0.9\paperheight}}{output_21_1.png}
    \end{center}
    { \hspace*{\fill} \\}
    
    \paragraph{When the second strain was initially tested with adults,
which responded very well to the flu virus with death rates close to 0,
even though their succeptibility rate is really high they can recover
quite
easily.}\label{when-the-second-strain-was-initially-tested-with-adults-which-responded-very-well-to-the-flu-virus-with-death-rates-close-to-0-even-though-their-succeptibility-rate-is-really-high-they-can-recover-quite-easily.}

    \begin{Verbatim}[commandchars=\\\{\}]
{\color{incolor}In [{\color{incolor}12}]:} \PY{n}{matplotlib}\PY{o}{.}\PY{n}{rcParams}\PY{p}{[}\PY{l+s+s1}{\PYZsq{}}\PY{l+s+s1}{figure.figsize}\PY{l+s+s1}{\PYZsq{}}\PY{p}{]} \PY{o}{=} \PY{p}{[}\PY{l+m+mi}{15}\PY{p}{,} \PY{l+m+mi}{7}\PY{p}{]}
         \PY{n}{plt}\PY{o}{.}\PY{n}{xlabel}\PY{p}{(}\PY{l+s+s1}{\PYZsq{}}\PY{l+s+s1}{Timestep}\PY{l+s+s1}{\PYZsq{}}\PY{p}{)}
         \PY{n}{plt}\PY{o}{.}\PY{n}{ylabel}\PY{p}{(}\PY{l+s+s1}{\PYZsq{}}\PY{l+s+s1}{Population size}\PY{l+s+s1}{\PYZsq{}}\PY{p}{)}
         \PY{n}{plt}\PY{o}{.}\PY{n}{title}\PY{p}{(}\PY{l+s+s1}{\PYZsq{}}\PY{l+s+s1}{Kids with soft virus strain}\PY{l+s+s1}{\PYZsq{}}\PY{p}{)}
         \PY{n}{plt}\PY{o}{.}\PY{n}{plot}\PY{p}{(}\PY{p}{[}\PY{n}{get\PYZus{}s}\PY{p}{(}\PY{l+m+mi}{1}\PY{p}{,} \PY{n}{x}\PY{p}{)} \PY{k}{for} \PY{n}{x} \PY{o+ow}{in} \PY{n+nb}{range}\PY{p}{(}\PY{l+m+mi}{100}\PY{p}{)}\PY{p}{]}\PY{p}{,} \PY{n}{label}\PY{o}{=}\PY{l+s+s1}{\PYZsq{}}\PY{l+s+s1}{Susceptible}\PY{l+s+s1}{\PYZsq{}}\PY{p}{)}
         \PY{n}{plt}\PY{o}{.}\PY{n}{plot}\PY{p}{(}\PY{p}{[}\PY{n}{get\PYZus{}i}\PY{p}{(}\PY{l+m+mi}{1}\PY{p}{,} \PY{l+m+mi}{1}\PY{p}{,} \PY{n}{x}\PY{p}{)} \PY{k}{for} \PY{n}{x} \PY{o+ow}{in} \PY{n+nb}{range}\PY{p}{(}\PY{l+m+mi}{100}\PY{p}{)}\PY{p}{]}\PY{p}{,} \PY{n}{label}\PY{o}{=}\PY{l+s+s1}{\PYZsq{}}\PY{l+s+s1}{Infected}\PY{l+s+s1}{\PYZsq{}}\PY{p}{)}
         \PY{n}{plt}\PY{o}{.}\PY{n}{plot}\PY{p}{(}\PY{p}{[}\PY{n}{get\PYZus{}r}\PY{p}{(}\PY{l+m+mi}{1}\PY{p}{,} \PY{l+m+mi}{1}\PY{p}{,} \PY{n}{x}\PY{p}{)} \PY{k}{for} \PY{n}{x} \PY{o+ow}{in} \PY{n+nb}{range}\PY{p}{(}\PY{l+m+mi}{100}\PY{p}{)}\PY{p}{]}\PY{p}{,} \PY{n}{label}\PY{o}{=}\PY{l+s+s1}{\PYZsq{}}\PY{l+s+s1}{Cured}\PY{l+s+s1}{\PYZsq{}}\PY{p}{)}
         \PY{n}{plt}\PY{o}{.}\PY{n}{legend}\PY{p}{(}\PY{p}{)}
\end{Verbatim}


\begin{Verbatim}[commandchars=\\\{\}]
{\color{outcolor}Out[{\color{outcolor}12}]:} <matplotlib.legend.Legend at 0x1117091d0>
\end{Verbatim}
            
    \begin{center}
    \adjustimage{max size={0.9\linewidth}{0.9\paperheight}}{output_23_1.png}
    \end{center}
    { \hspace*{\fill} \\}
    
    \paragraph{Even though children are highly succeptible to being
infected, in the case of the soft strain(flu) barely any children are
still infected after a short period of time and only arround half of the
population is still succeptible to getting the
virus.}\label{even-though-children-are-highly-succeptible-to-being-infected-in-the-case-of-the-soft-strainflu-barely-any-children-are-still-infected-after-a-short-period-of-time-and-only-arround-half-of-the-population-is-still-succeptible-to-getting-the-virus.}

    \subsubsection{Conclusion}\label{conclusion}

\subparagraph{Continous and discrete event system simulation has proven
to be a great way to understand the causes and effects of many "common"
events found all arround us. With a very interesting case such as virus
propagation we can find awnsers to how the population could react to
different settings created by a known pathogen and help us look for a
solution in order to prevent
loss.}\label{continous-and-discrete-event-system-simulation-has-proven-to-be-a-great-way-to-understand-the-causes-and-effects-of-many-common-events-found-all-arround-us.-with-a-very-interesting-case-such-as-virus-propagation-we-can-find-awnsers-to-how-the-population-could-react-to-different-settings-created-by-a-known-pathogen-and-help-us-look-for-a-solution-in-order-to-prevent-loss.}

\subparagraph{It is very interesting to see how the three
states(Succeptible, Infected, Recovered) for each case was affected by
the diferent probabilities of being hit by the virus and the strength of
it, or recovering from it which changed depending on the age of the
host.}\label{it-is-very-interesting-to-see-how-the-three-statessucceptible-infected-recovered-for-each-case-was-affected-by-the-diferent-probabilities-of-being-hit-by-the-virus-and-the-strength-of-it-or-recovering-from-it-which-changed-depending-on-the-age-of-the-host.}


    % Add a bibliography block to the postdoc
    
    
    
    \end{document}
